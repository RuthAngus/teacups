\section{Data}
\label{sec:data}

\newcommand{\ncomoving}{10,000}
\newcommand{\nfollow}{800}
\newcommand{\nfollowme}{363}
\newcommand{\ohpercent}{80}
\newcommand{\mypercent}{80}
\newcommand{\nconfirmed}{500}
\newcommand{\pairprob}{99.99\%}
\newcommand{\badpairprob}{50\%}

Here we use comoving pairs and groups of stars identified in the TGAS
catalogue by \citet{Oh2016}.
These stars do not have precise absolute ages: their absolute ages are
inferred by fitting isochrones to their photometric colours and parallaxes.
However their relative ages are very precise---they should be identical to
within a few million years as we assume that they formed at the same time from
the same molecular cloud.
We use the fact that these stars are coeval to test gyrochronology by
predicting their ages from their rotation periods.
We use the gyrochronal ages predicted for each star in the pair to infer the
intrinsic scatter the gyrochronology relations.

These comoving pairs were identified in the TGAS catalogue by identifying
stars with similar positions and proper motions by \citet{Oh2016}.
The probability that the stars were comoving was compared with the probability
that the stars were drawn from an isotropic, random distribution of
velocities.
In this way, \citet{Oh2016} identified \ncomoving\ comoving pair candidates.

Radial velocity (RV) follow-up of \nfollow\ comoving pair candidates was
performed in February 2017.
\ohpercent\% of these candidates were confirmed to have RVs consistent with
being a comoving pair.
Over seven nights, from the 26$^{th}$ of June to the 2${^nd}$ of July, we
obtained RVs of \nfollowme\ further comoving pair candidates which fall in
the \kepler\ field of view.
Since a small number of stars in the original \citet{Oh2016} catalogue fall in
the \kepler\ field, we used a slightly different sample with less stringent
false positive rates.
The original catalogue contains stars with a formal probability of being
comoving of \pairprob.
We used a catalogue of lower probability stars, specifically with a
\badpairprob\ formal probability of being a true comoving pair and selected
only stars within 10 parsecs of each other in order to reduce the false
positive rate.
\badpairprob\ may seem like a low probability, however with follow-up RV
observations we were able to identify many of the false positives in this
list.
We found \mypercent\% to be true comoving pairs based on their RVs.
We targeted stars in the \kepler\ field as these targets have ultra-precise
time-series photometry, from which we can measure rotation periods.
We selected the \nconfirmed\ confirmed comoving pairs to perform further
analysis.
