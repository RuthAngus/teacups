% \newcommand{\nconfirmed}{500}
\newcommand{\dantodo}[1]{{\color{blue}#1}}

\section{Method}
\label{sec:method}

\subsection{Rotation period measurement}

Since the \nconfirmed\ comoving stars identified in TGAS and confirmed with RV
follow-up we consider here are all \Kepler\ targets with precise light curves
covering around 4 years, we measured photometric rotation periods for these
stars.
We applied three different methods to the light curves: a Lomb-Scargle
periodogram method, an autocorrelation method and a Gaussian process method.
The exact implementations of the Lomb-Scargle and Gaussian process methods
were also used in the \citet{Angus2017} rotation period study.
In that work we found that the Gaussian process method produces slightly more
accurate rotation periods than the periodogram method and substantially more
accurate periods than the autocorrelation method for \kepler\ data.
However, we also found that each rotation period measurement method is very
sensitive to the exact implementations, for example any choices of heuristics
or priors affected the resulting rotation periods for an ensemble of light
curves.
For this reason we use all three methods here to ensure that our results are
not significantly altered by method choice.
We describe the implementation of each method below.

To measure periods using a Lomb-Scargle periodogram we first applied a
high-pass filter to the light curves to remove long-term trends.
We used the {\tt scipy} 3rd order Butterworth filter with a 35 day cut-off,
attenuating signals with periods greater than this threshold.
For each simulated light curve, we computed a LS periodogram\footnote{LS
periodograms were calculated using the {\tt scipy} Lomb-Scargle algorithm}
over a grid of 10,000 periods, evenly spaced in frequency, between 1 and 100
days.
We adopted the period of the highest peak in the periodogram as the measured
rotation period.
The uncertainties on the rotation periods were calculated using the
following equation for the standard deviation of the frequency
\citep{Horne1986, Kovacs1981}:
\begin{equation}
    \sigma_{\nu} = \frac{3\pi\sigma_N}{2N^{1/2}TA},
\end{equation}
where $A$ is the amplitude of the signal of highest power, $\sigma_N$ is the
variance of the time series, with the signal of highest power removed, $N$
is the number of observations and $T$ is the timespan of the data.
These formal uncertainties are only valid in the case that the noise is white,
the data are evenly sampled and there is only one signal present.
Since there are multiple signals present in these light curves, this formal
uncertainty is an underestimate of the true uncertainty.

\racomment{Add section on ACF method.}

The implementation of the Gaussian process method we use here is described and
discussed in detail in \citet{Angus2017}.
Given its success in \citet{Angus2017}, we chose to use the same kernel
function:
\begin{equation}
\label{eq:QP}
k_{i,j} = A \exp \left[-\frac{(x_i - x_j)^2}{2l^2} -
    \Gamma^2 \sin^2\left(\frac{\pi(x_i - x_j)}{P}\right) \right] + \sigma^2
    \delta_{ij},
\end{equation}
where $A$ is the covariance amplitude, $l$ is the length-scale of overall
covariance decay, $\Gamma$ is the parameter controlling
the intra-period variation (large $\Gamma$ permits more zero-crossings), $P$
is the rotation period and $\sigma$ is the additional white noise needed to
account for errorbar uncertainties.
The priors on each of these parameters takes a relatively simple functional
form, except for the period prior which is built from an initial
autocorrelation function guess.
These priors are fully described in \citet{Angus2017} and we will not go into
further detail here.

The rotation periods measured using these three methods are in good agreement:
N\% of periods agree to within 10\%.

\subsection{Age prediction}

Since we are assuming that the comoving stars are coeval we expect their
rotation periods to reflect this.
To quantify the rate of age-matches,

\subsection{Inferring intrinsic dispersion in the gyrochronology
relations}
