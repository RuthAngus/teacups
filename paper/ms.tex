\documentclass[useAMS, usenatbib, preprint, 12pt]{aastex}
\usepackage{cite, natbib}
\usepackage{float}
\usepackage{epsfig}
\usepackage{cases}
\usepackage[section]{placeins}
\usepackage{graphicx, subfigure}
\usepackage{color}
\usepackage{bm}

\newcommand{\naigrain}{333}
\newcommand{\nmcquillan}{100}
\newcommand{\kepexample}{1430163}
\newcommand{\kepexampleperiod}{4}
\newcommand{\aigrainexampleperiod}{20.8}
\newcommand{\Kepler}{{\it Kepler}}
\newcommand{\kepler}{\Kepler}
\newcommand{\corot}{{\it CoRoT}}
\newcommand{\Ktwo}{{\it K2}}
\newcommand{\ktwo}{\Ktwo}
\newcommand{\TESS}{{\it TESS}}
\newcommand{\tess}{{\it TESS}}
\newcommand{\LSST}{{\it LSST}}
\newcommand{\Wfirst}{{\it Wfirst}}
\newcommand{\SDSS}{{\it SDSS}}
\newcommand{\PLATO}{{\it PLATO}}
\newcommand{\Gaia}{{\it Gaia}}
\newcommand{\gaia}{{\it Gaia}}
\newcommand{\Teff}{$T_{\mathrm{eff}}$}
\newcommand{\teff}{$T_{\mathrm{eff}}$}
\newcommand{\FeH}{[Fe/H]}
\newcommand{\feh}{[Fe/H]}
\newcommand{\ie}{{\it i.e.}}
\newcommand{\eg}{{\it e.g.}}
\newcommand{\logg}{log \emph{g}}
\newcommand{\dnu}{$\Delta \nu$}
\newcommand{\numax}{$\nu_{\mathrm{max}}$}
\newcommand{\acfRMS}{9.27}
\newcommand{\pgramRMS}{1.25}
\newcommand{\mcmcRMS}{0.46}
\newcommand{\racomment}[1]{{\color{red}#1}}

\newcommand{\columbia}{1}
\newcommand{\simons}{2}
\newcommand{\pton}{3}
\newcommand{\nyu}{3}
\newcommand{\cca}{4}
\newcommand{\mpia}{5}
\newcommand{\cds}{6}
\newcommand{\uw}{7}

\begin{document}

\title{Testing gyrochronology with Gaia comoving pairs}

\author{%
    Ruth Angus\altaffilmark{\columbia, }\altaffilmark{\simons},
    Semyeong Oh\altaffilmark{\pton},
    Adrian Price-Whelan\altaffilmark{\pton},
    David W. Hogg\altaffilmark{\nyu, }\altaffilmark{\cca, }
    \altaffilmark{\mpia, }\altaffilmark{\cds},
    Daniel Foreman-Mackey\altaffilmark{\cca, }\altaffilmark{\uw},
    Matt Wilde\altaffilmark{\uw},
    Stephanie Douglas\altaffilmark{\columbia},
    David Kipping\altaffilmark{\columbia},
    Marcel Agueros\altaffilmark{\columbia} \&
    David Spergel\altaffilmark{\pton, }\altaffilmark{\cca}
}

\altaffiltext{\columbia}{Department of Astronomy, Columbia
University, NY, NY}
\altaffiltext{\simons}{Simons Fellow, RuthAngus@gmail.com}
\altaffiltext{\nyu}{Center for Cosmology and Particle Physics, New York
University, NY, NY}
\altaffiltext{\cca}{Center for Computational Astrophysics, Flatiron Institute,
NY, NY}
\altaffiltext{\mpia}{Max Planck Institute of Astronomy, Heidelberg, Germany}
\altaffiltext{\cds}{Center for Data Science, New York University, NY, NY}
\altaffiltext{\pton}{Princeton University}
\altaffiltext{\uw}{University of Washington}

\begin{abstract}
The precision of the gyrochronology method for inferring stellar ages is
    unknown outside of a few open clusters.
A small number of field stars with asteroseismic ages and photmetric rotation
periods, observed by \Kepler, show either a departure from a continuous
    magnetic braking regime, or a large scatter in their rotation periods at a
    given age.
Comoving pairs identified in the \Gaia\ DR1 TGAS catalogue reveal the
    precision and accuracy of gyrochronology at a range of ages.
They provide a bridge between clusters and isolated field stars.
We use the rotation periods of these comoving pairs to test the precision of
    gyrochronology.
We constrain the intrinsic scatter in the gyrochronology relations for these
    stars.
We find that a range of rotation periods appear present at each age.
\end{abstract}

\newcommand{\nastero}{40}
\newcommand{\meansep}{1}

\section{Introduction}

The rotational evolution of main sequence FGK stars via magnetic braking has
been studied since the late 1960s \citep[\eg][]{Weber, Skumanich}.
These stars appear to lose angular momentum via their magnetised stellar wind.
A convenient feature of this magnetic dynamo-driven angular momentum loss
causes rotation periods to converge to a single rotation-age-colour relation
after around 500 Myrs.
Observation of young clusters reveal a large spread in rotation periods at a
given age and mass, but this spread diminishes over time; older clusters show
a reduced spread.
The theoretical explanation for this convergence of rotation periods can be
understood from the rotation dependence of the angular momentum loss.
Rapid rotators lose angular momentum faster than slow rotators and 500 Myrs of
magnetic braking is sufficient to spin-down the rapid rotators to the same
rotation periods as their originally more slowly rotating fellow cluster
members.
After 500 Myrs, the current rotation period and colour, mass or effective
temperature of an FGK dwarf can be used to predict its age.
This process is called gyrochronology.

Many gyrochronology relations have been calibrated using stars with reliable
ages and rotation periods \citep[\eg][]{Barnes2007, Mamajek2008, Meibom2010,
Angus2015, Vansaders2016}.
The majority of these studies used young open clusters, lacking a sample of
old stars with reliable ages.
The \citet{Angus2015} and \citet{Vansaders2016} studies were able to use new
asteroseismic ages from the \kepler\ spacecraft to calibrate the gyrchronology
relations at late ages.
These studies found that the rotation periods of older stars diverged from the
expected power-law relation.
\citet{Vansaders2016} fit these data with a model in which magnetic braking
ceases after stars reach a critical Rossby number, $Ro$, the ratio of rotation
period to convective overturn time.
The Rossby number decreases with decreasing rotation period since stars spin
down, yet their convective overturn time, $\tau$, is mostly determined by
stellar mass and therefore remains constant.
It is the combination of rotation and convective motion in the outer
convective zone that produces the magnetic dynamo and Rossby number is
therefore a reasonable proxy for magnetic activity.
\citet{Vansaders2016} found the critical Rossby number to be 2.6.
As stars' ratio of rotation period to convective overturn time creep below
this number, magnetic braking appears to cease.
Since the works of \citet{Angus2015} and \citet{Vansaders2016}, further
evidence for a transitioning dynamo was discussed by \citet{...}.

The revised \citet{Vansaders2016} gyrochronology model indicates that
Solar-mass stars cease braking at around Solar age, higher mass stars cease
braking at earlier times and lower mass cease braking at later times.
This mass dependence is built into the Rossby number parameterisation of the
\citet{Vansaders2016} model and describes the observations well.
The \citet{Vansaders2016} model is able to reproduce the data, however these
data are sparse---there are only around \nastero\ FGK dwarfs with
asteroseismic ages and rotation periods---and the data sets used to calibrate
are heterogeneous and are subject to large selection and detection biases.
In order to confirm and refine the \citet{Vansaders2016} model, more data are
needed and specifically more ages of old stars with rotation periods will be
particularly valuable.
Such data are likely to be provided by the \tess, the Transiting Exoplanet
Survey Satellite.
Before those data become available however, we can use stars with ages derived
from other methods, not as precise as asteroseismology but numerous, to test
the gyrochronology models and characterise the intrinsic scatter in the
relations.

\subsection{The rotation period gap}
In addition to the transition in magnetic dynamo behaviour observed at old
ages by \citet{angus2015} and \citet{vansaders2016}, \kepler\ data revealed
another quirk of magnetic dynamo evolution that was discovered by
\citet{mcquillan2014}.
\citet{mcquillan2014} measured the rotation periods of more than 34,000
\kepler\ stars and showed a gap in the rotation period distribution of M
dwarfs at around 15 days.
Later, \citet{davenport2017} used \Gaia\ parallaxes to eliminate contaminating
dwarfs from the \citet{mcquillan2014} sample and revealed that this gap
extends into the FGK stars too.
Its presence can be explained by two scenarios: either a discontinuity in the
local star formation history around 600 Myrs ago is being reflected in the
rotation distribution (the gap appears to follow a 600 Myr isochrone, located
at shorter periods in hot stars and longer periods in cool stars) or it can be
explained by some rapid transition in magnetic dynamo behaviour at 600 Myrs.
This change in the magnetic field would have to result in a rapid,
super-Skumanich spin-down of stars at 600 Myrs.

\subsection{The origin of wide binaries}
The mean separation of the comoving pairs used in this analysis is \meansep\
parsec.
As such we do not call them wide binaries as their origin is unknown.
We suppose that they were binaries at some point in the past and have been
recently disrupted, however binaries have never previously been found at these
large separations.
The discovery of Kronos and Krious \citep{oh2017}, a comoving pair from
the \citep{oh2017} catalogue is an example of two chemically dissimilar
comoving stars.
The origin of their dramatically different chemical abundances is either due
to the recent ingestion of a planet by one of the stars, with some refactory
elements remaining on the stellar surface, or the stars had different birthing
environments.
The arguments rely on the strength of the common birth-site assumption: either
you assume the stars were born together and the different chemistry must
have arisen after birth, or you relax this assumption and the most likely
scenario might be that they were never chemical twins.
This same circular logic applies to our rotational project: in scenario one we
assume that these stars are coeval and use them to better understand the
precision and accuracy of gyrochronology, and in scenario two we relax this
assumption and use the similarity of their rotation periods to establish
whether they are coeval in the first place.
In a perfect world every member of a comoving pair will have the same
mass-adjusted rotation period as its partner and we can categorically state
that both gyrochronology is accurate {\it and} these stars are the same ages.
Inevitably there will be some noise and we will not be able to entirely
distinguish scenario one noise from scenario two noise.


\section{Data}
\label{sec:data}

\newcommand{\ncomoving}{10,000}
\newcommand{\nfollow}{800}
\newcommand{\nfollowme}{363}
\newcommand{\ohpercent}{80}
\newcommand{\mypercent}{80}
\newcommand{\nconfirmed}{500}
\newcommand{\pairprob}{99.99\%}
\newcommand{\badpairprob}{50\%}

Here we use comoving pairs and groups of stars identified in the TGAS
catalogue by \citet{Oh2016}.
These stars do not have precise absolute ages: their absolute ages are
inferred by fitting isochrones to their photometric colours and parallaxes.
However their relative ages are very precise---they should be identical to
within a few million years as we assume that they formed at the same time from
the same molecular cloud.
We use the fact that these stars are coeval to test gyrochronology by
predicting their ages from their rotation periods.
We use the gyrochronal ages predicted for each star in the pair to infer the
intrinsic scatter the gyrochronology relations.

These comoving pairs were identified in the TGAS catalogue by identifying
stars with similar positions and proper motions by \citet{Oh2016}.
The probability that the stars were comoving was compared with the probability
that the stars were drawn from an isotropic, random distribution of
velocities.
In this way, \citet{Oh2016} identified \ncomoving\ comoving pair candidates.

Radial velocity (RV) follow-up of \nfollow\ comoving pair candidates was
performed in February 2017.
\ohpercent\% of these candidates were confirmed to have RVs consistent with
being a comoving pair.
Over seven nights, from the 26$^{th}$ of June to the 2${^nd}$ of July, we
obtained RVs of \nfollowme\ further comoving pair candidates which fall in
the \kepler\ field of view.
Since a small number of stars in the original \citet{Oh2016} catalogue fall in
the \kepler\ field, we used a slightly different sample with less stringent
false positive rates.
The original catalogue contains stars with a formal probability of being
comoving of \pairprob.
We used a catalogue of lower probability stars, specifically with a
\badpairprob\ formal probability of being a true comoving pair and selected
only stars within 10 parsecs of each other in order to reduce the false
positive rate.
\badpairprob\ may seem like a low probability, however with follow-up RV
observations we were able to identify many of the false positives in this
list.
We found \mypercent\% to be true comoving pairs based on their RVs.
We targeted stars in the \kepler\ field as these targets have ultra-precise
time-series photometry, from which we can measure rotation periods.
We selected the \nconfirmed\ confirmed comoving pairs to perform further
analysis.


\newcommand{\nconfirmed}{500}
\newcommand{\dantodo}[1]{{\color{blue}#1}}

\section{Method}
\label{sec:method}

\subsection{Rotation period measurement}

Since the \nconfirmed\ comoving stars identified in TGAS and confirmed with RV
follow-up we consider here are all \Kepler\ targets, we were able to attempt
to measure the rotation periods from the \kepler\ light curves of these stars.
We applied three different methods to the light curves: a Lomb-Scargle
periodogram method, an autocorrelation method and a Gaussian process method.
The exact implementations of the Lomb-Scargle and Gaussian process methods
were also used in the \citet{Angus2017} rotation period study.
In this previous work we found that the Gaussian process method produces
slightly more accurate rotation periods than the periodogram method and
substantially more accurate periods than the autocorrelation method.
However, we also found that each rotation period measurement method is very
sensitive to the exact implementations, for example any choices of heuristics
or priors affected the resulting rotation periods for an ensemble of light
curves.
For this reason we use all three methods here to ensure that our results are
not significantly altered by method choice.
We describe the implementation of each method below.

To measure periods using a Lomb-Scargle periodogram we first applied a
high-pass filter to the light curves to remove long-term trends.
We used the {\tt scipy} 3rd order Butterworth filter with a 35 day cut-off,
attenuating signals with periods greater than this threshold.
For each simulated light curve, we computed a LS periodogram\footnote{LS
periodograms were calculated using the {\tt scipy} Lomb-Scargle algorithm}
over a grid of 10,000 periods, evenly spaced in frequency, between 1 and 100
days.
We adopted the period of the highest peak in the periodogram as the measured
rotation period.
The uncertainties on the rotation periods were calculated using the
following equation for the standard deviation of the frequency
\citep{Horne1986, Kovacs1981}:
\begin{equation}
    \sigma_{\nu} = \frac{3\pi\sigma_N}{2N^{1/2}TA},
\end{equation}
where $A$ is the amplitude of the signal of highest power, $\sigma_N$ is the
variance of the time series, with the signal of highest power removed, $N$
is the number of observations and $T$ is the timespan of the data.
These formal uncertainties are only valid in the case that the noise is white,
the data are evenly sampled and there is only one signal present.
Since there are multiple signals present in these light curves, this formal
uncertainty is an underestimate of the true uncertainty.

\racomment{Add section on ACF method.}

The implementation of the Gaussian process method we use here is described and
discussed in detail in \citet{Angus2017}.
Given its success in \citet{Angus2017}, we chose to use the same kernel
function:
\begin{equation}
\label{eq:QP}
k_{i,j} = A \exp \left[-\frac{(x_i - x_j)^2}{2l^2} -
    \Gamma^2 \sin^2\left(\frac{\pi(x_i - x_j)}{P}\right) \right] + \sigma^2
    \delta_{ij},
\end{equation}
where $A$ is the covariance amplitude, $l$ is the length-scale of overall
covariance decay, $\Gamma$ is the parameter controlling
the intra-period variation (large $\Gamma$ permits more zero-crossings), $P$
is the rotation period and $\sigma$ is the additional white noise needed to
account for errorbar uncertainties.
The priors on each of these parameters takes a relatively simple functional
form, except for the period prior which is built from an initial
autocorrelation function guess.
These priors are fully described in \cite{Angus2017} and we will not go into
further detail here.

The rotation periods measured using these three methods are in good agreement:
N\% of periods agree to within 10\%.

\subsection{Age prediction}

Since we are assuming that the comoving stars are coeval we expect their
rotation periods to reflect this.
To quantify the rate of age-matches,

\subsection{Inferring intrinsic dispersion in the gyrochronology
relations}


\section{Results and Discussion}
\label{sec:results_and_discussion}

\subsection{The rotation period gap}
In addition to the transition in magnetic dynamo behaviour observed at old
ages by \citet{angus2015} and \citet{vansaders2016}, \kepler\ data revealed
another quirk of magnetic dynamo evolution that was discovered by
\citet{mcquillan2014}.
\citet{mcquillan2014} measured the rotation periods of more than 34,000
\kepler\ stars and revealed a gap in the rotation period distribution of M
dwarfs at around 15 days.
Later, \citet{davenport2017} used \Gaia\ parallaxes to eliminate contaminating
dwarfs from the \citet{mcquillan2014} sample, showing that this gap extends
into the FGK stars too.
Its presence can be explained by two scenarios: either a discontinuity in the
local star formation history around 600 Myrs ago is being reflected in the
rotation distribution (the gap appears to follow a 600 Myr isochrone, located
at shorter periods in hot stars and longer periods in cool stars) or it can be
explained by some rapid transition in magnetic dynamo behaviour at 600 Myrs.
This change in the magnetic field would have to result in a rapid,
super-Skumanich spin-down of stars at 600 Myrs in order to explain the
observations.
We searched for


\include{conclusion}

% acknowledgements
This research was funded by the Simons Foundation.
This work is based on observations obtained at the MDM Observatory, operated
by Dartmouth College, Columbia University, Ohio State University, Ohio
University, and the University of Michigan.
Some of the data presented in this paper were obtained from the Mikulski
Archive for Space Telescopes (MAST).
STScI is operated by the Association of Universities for Research in
Astronomy, Inc., under NASA contract NAS5-26555.
Support for MAST for non-HST data is provided by the NASA Office of Space
Science via grant NNX09AF08G and by other grants and contracts.
This paper includes data collected by the Kepler mission. Funding for the
Kepler mission is provided by the NASA Science Mission directorate.

\end{document}

% \bibliographystyle{plainnat}
% \bibliography{teacups}
% \end{document}
