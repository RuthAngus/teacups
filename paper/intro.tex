\newcommand{\nastero}{40}
\newcommand{\meansep}{1}

\section{Introduction}

The rotational evolution of main sequence FGK stars via magnetic braking has
been studied since the late 1960s \citep[\eg][]{Weber, Skumanich}.
These stars appear to lose angular momentum via their magnetised stellar wind.
A convenient feature of this magnetic dynamo-driven angular momentum loss
causes rotation periods to converge to a single rotation-age-colour relation
after around 500 Myrs.
Observation of young clusters reveal a large spread in rotation periods at a
given age and mass, but this spread diminishes over time; older clusters show
a reduced spread.
The theoretical explanation for this convergence of rotation periods can be
understood from the rotation dependence of the angular momentum loss.
Rapid rotators lose angular momentum faster than slow rotators and 500 Myrs of
magnetic braking is sufficient to spin-down the rapid rotators to the same
rotation periods as their originally more slowly rotating fellow cluster
members.
After 500 Myrs, the current rotation period and colour, mass or effective
temperature of an FGK dwarf can be used to predict its age.
This process is called gyrochronology.

Many gyrochronology relations have been calibrated using stars with reliable
ages and rotation periods \citep[\eg][]{Barnes2007, Mamajek2008, Meibom2010,
Angus2015, Vansaders2016}.
The majority of these studies used young open clusters, lacking a sample of
old stars with reliable ages.
The \citet{Angus2015} and \citet{Vansaders2016} studies were able to use new
asteroseismic ages from the \kepler\ spacecraft to calibrate the gyrchronology
relations at late ages.
These studies found that the rotation periods of older stars diverged from the
expected power-law relation.
\citet{Vansaders2016} fit these data with a model in which magnetic braking
ceases after stars reach a critical Rossby number, $Ro$, the ratio of rotation
period to convective overturn time.
The Rossby number decreases with decreasing rotation period since stars spin
down, yet their convective overturn time, $\tau$, is mostly determined by
stellar mass and therefore remains constant.
It is the combination of rotation and convective motion in the outer
convective zone that produces the magnetic dynamo and Rossby number is
therefore a reasonable proxy for magnetic activity.
\citet{Vansaders2016} found the critical Rossby number to be 2.6.
As stars' ratio of rotation period to convective overturn time creep below
this number, magnetic braking appears to cease.
Since the works of \citet{Angus2015} and \citet{Vansaders2016}, further
evidence for a transitioning dynamo was discussed by \citet{...}.

The revised \citet{Vansaders2016} gyrochronology model indicates that
Solar-mass stars cease braking at around Solar age, higher mass stars cease
braking at earlier times and lower mass cease braking at later times.
This mass dependence is built into the Rossby number parameterisation of the
\citet{Vansaders2016} model and describes the observations well.
The \citet{Vansaders2016} model is able to reproduce the data, however these
data are sparse---there are only around \nastero\ FGK dwarfs with
asteroseismic ages and rotation periods---and the data sets used to calibrate
are heterogeneous and are subject to large selection and detection biases.
In order to confirm and refine the \citet{Vansaders2016} model, more data are
needed and specifically more ages of old stars with rotation periods will be
particularly valuable.
Such data are likely to be provided by the \tess, the Transiting Exoplanet
Survey Satellite.
Before those data become available however, we can use stars with ages derived
from other methods, not as precise as asteroseismology but numerous, to test
the gyrochronology models and characterise the intrinsic scatter in the
relations.

\subsection{The rotation period gap}
In addition to the transition in magnetic dynamo behaviour observed at old
ages by \citet{angus2015} and \citet{vansaders2016}, \kepler\ data revealed
another quirk of magnetic dynamo evolution that was discovered by
\citet{mcquillan2014}.
\citet{mcquillan2014} measured the rotation periods of more than 34,000
\kepler\ stars and showed a gap in the rotation period distribution of M
dwarfs at around 15 days.
Later, \citet{davenport2017} used \Gaia\ parallaxes to eliminate contaminating
dwarfs from the \citet{mcquillan2014} sample and revealed that this gap
extends into the FGK stars too.
Its presence can be explained by two scenarios: either a discontinuity in the
local star formation history around 600 Myrs ago is being reflected in the
rotation distribution (the gap appears to follow a 600 Myr isochrone, located
at shorter periods in hot stars and longer periods in cool stars) or it can be
explained by some rapid transition in magnetic dynamo behaviour at 600 Myrs.
This change in the magnetic field would have to result in a rapid,
super-Skumanich spin-down of stars at 600 Myrs.

\subsection{The origin of wide binaries}
The mean separation of the comoving pairs used in this analysis is \meansep\
parsec.
As such we do not call them wide binaries as their origin is unknown.
We suppose that they were binaries at some point in the past and have been
recently disrupted, however binaries have never previously been found at these
large separations.
The discovery of Kronos and Krious \citep{oh2017}, a comoving pair from
the \citep{oh2017} catalogue is an example of two chemically dissimilar
comoving stars.
The origin of their dramatically different chemical abundances is either due
to the recent ingestion of a planet by one of the stars, with some refactory
elements remaining on the stellar surface, or the stars had different birthing
environments.
The arguments rely on the strength of the common birth-site assumption: either
you assume the stars were born together and the different chemistry must
have arisen after birth, or you relax this assumption and the most likely
scenario might be that they were never chemical twins.
This same circular logic applies to our rotational project: in scenario one we
assume that these stars are coeval and use them to better understand the
precision and accuracy of gyrochronology, and in scenario two we relax this
assumption and use the similarity of their rotation periods to establish
whether they are coeval in the first place.
In a perfect world every member of a comoving pair will have the same
mass-adjusted rotation period as its partner and we can categorically state
that both gyrochronology is accurate {\it and} these stars are the same ages.
Inevitably there will be some noise and we will not be able to entirely
distinguish scenario one noise from scenario two noise.
