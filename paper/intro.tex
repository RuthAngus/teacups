\newcommand{\nastero}{40}
\newcommand{\meansep}{1}

\section{Introduction}

The slow decay of rotational velocities in main sequence, FGK stars via
magnetic braking has been established since the late 1960s
\citep[\eg][]{Weber, Skumanich}.
It is widely accepted that these stars lose angular momentum via their
magnetized stellar wind and the rate of angular momentum loss is a function of
stellar mass \racomment{CITATIONS}.
Observations of young clusters indicate that the rotation periods of FGK dwarfs
converge to a single rotation-age-mass relation after around 500 Myrs
\racomment{CITATIONS}.
A large range of rotation periods at a given mass is observed in the youngest
star clusters, but this spread diminishes over time and older clusters show a
reduced spread.
The explanation for this convergence of rotation periods is provided by the
rotational dependence of angular momentum loss.
Rapid rotators lose angular momentum faster than slow rotators and 500 Myrs of
magnetic braking is sufficient to spin-down the rapid rotators to the same
rotation periods as their originally more slowly rotating fellow cluster
members.
The gyrochronology method uses rotation period and mass, or mass proxy, to
predict the age of a star.

Many gyrochronology relations have been calibrated using stars with reliable
ages and rotation periods \citep[\eg][]{Barnes2007, Mamajek2008, Meibom2010,
Angus2015, Vansaders2016}.
The majority of these studies used young open clusters, lacking a sample of
old stars with reliable ages.
The \citet{Angus2015} and \citet{Vansaders2016} studies incorporated new
asteroseismic ages from the \kepler\ spacecraft \citep{borucki2010} to
calibrate the gyrchronology relations at late ages.
These studies found that the rotation periods of older stars diverged from the
expected power-law relation.
\citet{Vansaders2016} fit these data with a new model in which magnetic
braking ceases after stars reach a critical Rossby number, $Ro$, the ratio of
rotation period to convective overturn time.
The Rossby number decreases with decreasing rotation period since stars spin
down, yet their convective overturn time, $\tau$, is mostly determined by
stellar mass and therefore remains constant.
It is the combination of rotation and convective motion in the outer
convective zone that produces the magnetic dynamo and Rossby number is
therefore a reasonable proxy for magnetic activity.
\citet{Vansaders2016} found the critical Rossby number to be 2.6
\racomment{Uncertainty?}; magnetic braking ceases in stars with $Ro < 2.6$.
\racomment{Add Metcalfe paragraph.}
The revised \citet{Vansaders2016} gyrochronology model indicates that
Solar-mass stars cease braking at around Solar age, higher mass stars cease
braking at earlier times and lower mass cease braking at later times.
This mass dependence is built into the Rossby number parameterisation of the
\citet{Vansaders2016} model and describes the observations well.
The \citet{Vansaders2016} model is able to reproduce the data, however these
data are sparse---there are only around \nastero\ FGK dwarfs with
asteroseismic ages and rotation periods---and the data sets used for
calibration are heterogeneous and subject to selection and detection biases.
More data are needed in order to confirm and refine the \citet{Vansaders2016}
model, specifically more precise ages of old stars with rotation periods will
be particularly valuable.
Such data are likely to be provided by \tess, the Transiting Exoplanet Survey
Satellite \citep{ricker2015}.
Before those data become available however, we can use comparative ages to
test the gyrochronology models and characterize the intrinsic scatter in the
relations, such as those provided by binaries and clusters.

\subsection{The origin of wide binaries.}
\racomment{Some background on binary formation and longevity.}

\subsection{The origin of wide binaries}
The \citet{oh2016} catalogue of wide binaries discovered with \gaia\ data
provide a sample of widely separated pairs and groups of stars.
\citet{oh2016} discovered comoving pairs of stars as widely separated as 10
kiloparsecs, and they likely would have discovered pairs with even greater
separations, had they relaxed the strict 10 kpc search cut off.
\citet{andrew2017} demonstrate that there is a high false-positive rate of
binary stars (stars that are coincidentally comoving) at separations larger
than ....
However, it is likely that a fraction of the largest separation binaries are
indeed real.
The mean separation of the comoving pairs used in this analysis is \meansep\
parsec.
Kronos and Krios \citep{oh2017}, a pair of comoving stars in the
\citep{oh2016} catalog is an example of two chemically dissimilar comoving
stars.
The origin of their different chemical abundances is either due to the recent
ingestion of a planet by one of the stars, with some refactory elements
remaining on the stellar surface, or the stars had different birthing
environments.
The arguments rely on the strength of the common birth-site assumption: either
you assume the stars were born together and the different chemistry must
have arisen after birth, or you relax this assumption and the most likely
scenario might be that they were never chemical twins.
This same circular logic applies to our rotational project: in scenario one we
assume that these stars are coeval and use them to better understand the
precision and accuracy of gyrochronology, and in scenario two we relax this
assumption and use the similarity of their rotation periods to establish
whether they are coeval in the first place.
In a perfect world every member of a comoving pair will have the same
mass-adjusted rotation period as its partner and we can categorically state
that both gyrochronology is accurate {\it and} these stars are the same ages.
Inevitably there will be some noise and we will not be able to entirely
distinguish scenario one noise from scenario two noise.
