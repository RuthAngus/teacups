\section{Results and Discussion}
\label{sec:results_and_discussion}

\subsection{The rotation period gap}
In addition to the transition in magnetic dynamo behaviour observed at old
ages by \citet{angus2015} and \citet{vansaders2016}, \kepler\ data revealed
another quirk of magnetic dynamo evolution that was discovered by
\citet{mcquillan2014}.
\citet{mcquillan2014} measured the rotation periods of more than 34,000
\kepler\ stars and revealed a gap in the rotation period distribution of M
dwarfs at around 15 days.
Later, \citet{davenport2017} used \Gaia\ parallaxes to eliminate contaminating
dwarfs from the \citet{mcquillan2014} sample, showing that this gap extends
into the FGK stars too.
Its presence can be explained by two scenarios: either a discontinuity in the
local star formation history around 600 Myrs ago is being reflected in the
rotation distribution (the gap appears to follow a 600 Myr isochrone, located
at shorter periods in hot stars and longer periods in cool stars) or it can be
explained by some rapid transition in magnetic dynamo behaviour at 600 Myrs.
This change in the magnetic field would have to result in a rapid,
super-Skumanich spin-down of stars at 600 Myrs in order to explain the
observations.
We searched for
